\section{Problem analysis}
\todo{
\begin{itemize}
\item Coarse problem analysis Describing the problem and goals in more
detail, briefly analysing the challenges, pointing out possible solutions.
\item Problem specification and analysis Detailed specification of the
problem (and goals). This might need to introduce new notation in order
to arrive at a precise description. If relevant, make state-of-the-art
solutions precise and compare.
\item Goals and success criteria Describe the goals and when a goal is
reached.
\end{itemize}
}

The problem with making cooperative agents in hanabi is mainly due to the fact that each agent does not know its own hand, and this lack of knowledge leads to a huge number of possible state-spaces. 
A naive approach to represent this lack of knowledge is simply to generating all possible hands given the information the agent has.
This, I would argue, is pretty straight-forward to work with, but if done improper will lead to intolerable waiting times.
This in combination with a modal logic approach can easily lead to a huge generation of combinations. 
For example let us denote a game played by two players Alice and Bob. 
If Alice sees Bobs cards (and the discard pile, and hanabi pile), then Alice can generate the set of possible hands $\mathcal{H}$, but in order to also represent the knowledge of Bob, Alice would have to for each $h \in \mathcal{H}$ generate a new set of possible hands for Bob.
A different approach is simply to store the hints and shown cards at the agents, and at some proper time -- when a significant amount of simple knowledge has been accumulated -- produce the set $\mathcal{H}$ and the relevant sets for the other players.


