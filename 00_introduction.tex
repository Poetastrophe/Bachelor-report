\section{Introduction}

Hanabi (meaning \emph{fireworks} in Japanese) is a cooperative card-game designed by Antoine Bauza \cite{BGGHanabi}. 
In the game each player has a set of cards in their hand and the players must cooperate in order to play the cards in a specific sequence to achieve the highest possible score of 25 points.
The main obstacle for the players is that a player cannot see their own hand, but can see the other players' hands. 
Despite this limitation each player must play cards from their own hand and the only way in which they can know which card to play is based on specific hints and counting the cards. 
Since what a player can know for sure is quite restricted, each player, in practice, will have to guess the intention of the other player i.e. have a theory of mind. 
Since it is a game with imperfect knowledge, as well as good strategies have some theory-of-mind in place, it has sparked some notable interest in the AI research \cite{DeepmindAndOthers}. 

In this thesis I will focus on \emph{self-play} i.e. only AI agents will have to cooperate in order to get as high a score as possible, as opposed to \emph{cross-play} where some or most of the players are actually humans that try to cooperate with one or more agents.  

Solutions to the problem includes hat-guessing strategies \cite{CoxEtAl2015}, Cox et al develop two strategies, one for recommending moves to other players, denoted recommendation strategy, and one for increasing other players' knowledge of what cards they currently hold, denoted information strategy. 
A hat-guessing strategy utilizes the fact that any legal move in the game can be interpreted as an encoding about some information or recommendation.
This encoding can then be decoded by each individual, giving rise to some information or recommended action. 
Such strategies have proven effective for Hanabi, with \cite{CoxEtAl2015} getting an average of $23.00$ points for their recommendation strategy and $24.68$ points for their information strategy. 
Looking into these strategies in detail, I see that these strategies are not only effective, but also efficient, because it is a very small amount of data each agent has to keep track of, and deducing the encoding and updating auxiliary data is quite trivial for a computer.
Other solutions utilize Dynamic Epistemic Logic \cite{EgerAndMartens17}, which can very declaratively specify goals and try to find shortest paths in order to satisfy these goals. 
There are also machine-learning oriented solutions to this problem \cite{hu2021otherplay}, that nowadays seem to be on-par with the rule-based ones achieving average score of $24.09$.

In this thesis I will describe my solution based on principles of Dynamic Epistemic Logic, with emphasis on knowledge about ones hand, in order to make a group of agents work together and attempt to score as many points as possible in Hanabi. 

The choice of implementation language is Zig \cite{Ziglang}, a new C-like language, which offers a lot of low-level control to the programmer, making it possible to create some highly-optimal data-structures as well as well as making some types of optimizations possible that would be hard to achieve with languages working on a higher level of abstraction.



\subsection{Instructions on how to run the code}
\todo[inline]{later}
You will need the Zig compiler in order to run the code. Navigate to this page \url{https://ziglang.org/download/} and download a version 0.10.1, matching your computer's architecture.

Once you have the compiler, you can navigate to the folder {\tt AgentsAndGameSimulator} and run the command: {\tt \$ zig build run-agents -Drelease-fast=true}

The {\tt Drelease-fast=true} is optional, but makes sure it is compiled with optimizations.



\subsection{Hanabi setup and rules}
For completion I will give a short summary of the rules in Hanabi, which will also serve as terminology later on in the report.

Hanabi consists of a \emph{Deck} of 50 cards, 4 \emph{black tokens}, 8 \emph{blue tokens}.  
Of the cards there are 5 \emph{suits} and 5 \emph{values}. 
The suits are red, green, blue, white, yellow. 
The values are 1 through 5. 
For each suit there is three cards of value 1, two cards of value 2, two cards of value 3, two cards of value 4 and one card of value 5.
The number of cards in each player's hand depends on the number of players. If there are 3 players or less, they will have 5 cards each. 
Otherwise they will have 4 cards each.

The goal of the game is to play add as many cards as possible to the \emph{colour piles}. 
There are 5 colour piles, one for each suit. 
You gain a point for each card in the colour piles, with a maximum of 25 points.

A player can spent her turn on one of the following actions

\begin{itemize}
\item Play a card
\item Give a hint
\item Discard a card
\end{itemize}

\paragraph{Playing}
When playing a card you attempt to add a card to one of the colour piles.
You can add a card $c$ to a pile of identical suit $p$ if either, 

\begin{itemize}	
\item $p$ is an empty colour pile and $c$ is of value 1.
\item $p$ is non-empty and $c$'s value is exactly one greater than the current card on top of the pile.
\end{itemize}

If a card does not add to a colour pile then it is put in the \emph{discard pile} and then the game removes 1 black token. 
If there are no black tokens the game ends immediately.  

\paragraph{Hinting}
If a player chooses to hint, he can hint any other player. 
Hints are restricted, so you can either hint about cards matching a suit or cards matching a value. 
Imagine a hand of following configuration: ((red,1),(blue,3),(red,2),(yellow,3)), each having respective indices 0 through 3. 
Then if you hint "cards of value 3", then you have to give the positions of both (blue,3) and (yellow,3) i.e. index 1 and 3. 
And if you give the hint "cards of red suit" then you have to give the position of both (red,1) and (red,2) i.e. index 0 and 2. 
You cannot give an "empty hint" like "there are no green cards".

When giving a hint it removes 1 blue token, if you cannot do that, you cannot give the hint.

\paragraph{Discarding}
A player can discard any card on their hand, and it will result in the card going in the discard pile and it adds 1 blue token (and can only be done if there are less than 8 blue tokens left).

\paragraph{End of the game}
The game ends immediately if there are no black tokens left. 
The game also ends if the last card from the deck has been drawn, in which case every player can play an additional round (including the player who drew the last card) and then the game ends.

After the game ends the score is decided by counting how many cards there is in the colour piles.




