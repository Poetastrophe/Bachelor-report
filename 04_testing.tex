\section{Testing}
% \ninanotes{
% \begin{itemize}
% \item Testing
% \item Describe which kinds of tests are used and why.
% \item Document the results.
% \item Document whether (early) testing has lead to changes in design or
% implementation.
% \end{itemize}
% }

\todo[inline]{I have to stress WHY I used these tests}
\subsection{Unit testing and integration testing}
I did some unit-testing in the project, which is directly integrated with the Zig language, in the form that you simply write the unit tests in the source files themselves, under a "test" environment. 
Things with a few dependencies are easy to verify and unit-tests give confidence to that these basic methods/classes can be relied on. These simple components are the {\tt CardSet} class, the permutation and distinct combination methods, or printing methods. 

Unit-testing becomes becomes infeasible when you have to combine and inspect more advanced properties.
For instance, when testing whether the generated worlds are correct, I would have to compare it to something much simpler.
In this case I tried to adapt my methods in order to make it match the three wise men problem, and then I would visually inspect it myself to see whether it was correct, this is done in the file {\tt src/multi\_agent\_solvers/agent.zig} in the test called "Three wise men simulation". So simple tests like that functioned as an integration test.

Unit-testing is pretty crucial when working with manual memory management languages like Zig, since the risk of orphaned memory is much more severe, and Zig facilitates prevention of memory leaks by detecting this in its unit tests. 


\subsection{Early feasibility test for the world generation method}
After implementing the table encoding for a world, I decided to make a feasibility test, whether it was realistic generating the possible worlds for Hanabi.
The test is in {\tt src/bigtest1.zig}.
I generate 20 random initial rounds to see how much space and time each one uses. The most amount of space used by a model for a round was 11.27 GB, and the least space was 3.82 GB.
The most amount of time spent on generating a model was 14.3 seconds, and the least time was 4.5 seconds.
These numbers are well within the constraints given by Problem Analysis chapter and therefore I decided that it would be feasible to continue with the current solution.
In this test I did not take into account how much time might actually be spent on querying and removing worlds, for more details on that see section \ref{sec:testing-kripkestructure-class}.

\subsection{Testing my game implementation by making an interactive interface}
I wanted to test the {\tt Game} class, to see if anything unexpected happened if some specific sequence of action occurs.
In order to do this, instead of manually making some unit tests, I implemented a TUI interface of the game and tried various things in order to "break" the {\tt Game} implementation.
The TUI interface is in {\tt src/cli\_simulation\_runner.zig}.

\todo[inline]{Make the program have different compile arguments, in order to facilitate, like, how to run the runner? etc}
\todo[inline]{Make install instructions on zig so that the user can run my program.}

\subsection{Ad hoc white-box testing system}
Some things I wanted to continually inspect throughout the project, and some of these things were related to the internal behaviour of the methods.
This is a type of white-box testing, where I inspect the internal structure of the methods.
In order to do this, without sacrificing performance from the methods themselves, I utilized the fact that Zig can optimize code-blocks away, that it knows at compile-time will not be executed.

This is done using a file that contain some global-accessible flags: {\tt src/multi\_agent\_solvers/globals\_test.zig}.

So what I mean by "it optimizes code-blocks away" I will illustrate with the pseudo-code at \ref{code:statically-removing-branches}. So we have some complicated function which does a lot of work on the {\tt kripkestructure}, and I want to iterate through the elements, to verify that it does it correctly. 
The iteration happens in the if-statement block. 
This block will be removed at compile-time if it is known that {\tt globals.check\_kripkestructure=false}, in this way I can enable it when I have a need for it.

\begin{verbbox}
function some_complicated_function(game, kripkestructure):
    // modifies kripkestructure in some way
    if(globals.check_kripkestructure){
        for each element in kripkestructure{
            print(element)
        }
    }
    // function continues to do other work
}
\end{verbbox}
{\centering
\fbox{\theverbbox}
\captionof{Code}{An example of how globals are used for testing}\par
\label{code:statically-removing-branches}
}

\subsection{Average score achieved by my product}
The average score achieved through simulating 20 random games is 14.2, with lowest score of 12, and highest score of 17, which means that the program is far from being as good as many existing solutions.

\subsection{Testing the time spent by {\tt KripkeStructure} class} \label{sec:testing-kripkestructure-class}
The bulk of the work done by the program is {\tt KripkeStruture.init(...)} and {\tt Agent.updateCardStates(...)}.
I will here mainly focus on the {\tt Kripkestructure} generation, and also test the time of {\tt KripkeStructure.deinit()}, because I think that there might be a lot of time spent deallocating the structure as well.

The {\tt KripkeStructure.init(...)} works in three steps
\begin{enumerate}
	\item Generate all possible worlds based on the visible cards (hints are not used)
	\item Eliminate all possible worlds in which they are in contradiction with the hints.
\end{enumerate}

The {\tt KripkeStructure.deinit(...)} method simply iterates through the structure and calls deinit on all substructures.

The reason that I examined these things was to 1) judge the efficiency of the distinct combination generator, 2) see if there was something to be gained from using a different type of allocator, because I know that I allocate and deallocate a lot of elements, 3) and if the method of removing worlds is actually efficient.

In 5 random games, it took 57.3 minutes to play all the games, for which it took 56.9 minutes for all the {\tt KripkeStructure.init(...)}, {\tt KripkeStructure.deinit()} and {\tt KripkeStructure.remove\_worlds\_based\_on\_hints(...)} calls.
Of the 56.9 minutes, each call took the following percentage of time: $24.5\%$ of the time is spent on generating the distinct combinations, $1.2\%$ of the time is spent on deallocating space and $74.2\%$ is spent on removing worlds based on hints.

I think the removal of worlds takes so much time, is that I have not optimized for identical permutations.
So no matter how many hints there are on a hand, it will generate all $k!$ permutations.



