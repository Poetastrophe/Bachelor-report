\section{Conclusion}
% \ninanotes{
% \begin{itemize}
% \item Conclusion
% \item What has been reached, how does the product work?
% \item What was not achieved or achieved exceeding the plan?
% \item Did you learn something new, techniques, socially?
% \item If there was more time . . .
% \item Personally, I . . . . . .
% \end{itemize}
% }

I have made a product based on a modified model of \SfiveN{}, denoted sub-\SfiveN{}, in which each agent can deduce a big set of possibilities for the other agents (and itself), and it is able to look through each possible world to extract some property about each world.
The average score is 14.2 which is pretty far from many existing solutions.
The memory-layout of sub-\SfiveN{} in the case of Hanabi, is quite simple, given that it is implemented using a 3-dimensional array, and each world spends only 25 bytes of memory, which are great for query algorithms that look through each world sequentially.  
Further space was saved by making sure that each world stores a multiset of cards, instead of a list of cards, while still retaining the ability to answer queries about specific cards on a players hand.

If I had more time I would look into how public knowledge among the agents could be deduced from sub-\SfiveN{} model.
Given that public knowledge is very useful for strategies like the information strategy in \cite{CoxEtAl2015}.

If I had more time, it would be interesting to see which strategies would work great in conjunction with the sub-\SfiveN{} model.
For instance the two rule-based strategies described in \cite{CoxEtAl2015}, I would argue, works completely orthogonally to the sub-\SfiveN{} model solution, which means that both of these strategies could be combined with sub-\SfiveN{} model in order to make better decisions.
In the case of the information strategy, then the information gained by encoding the moves could further modify each agent's model and thereby gain more information by using such a strategy. 
Similarly if the recommendation strategy says one of your cards is "playable" then you can eliminate all worlds for which that card is not playable.

If there was more time I would also like to look into how to optimize some aspects of the code, because right now there are a lot of redundant permutations for each hand that are being checked, which makes it spent more time than necessary trying to go through the possible worlds.

If I had more time, I would have made a more in-depth comparison of the data-representations of the worlds described in section \ref{sec:representing-a-world}, and see how they might have affected the performance.

\todo[inline]{remember to also note the average time spent on each round, and the maximal amount of time and minumum}


In this project I learned a lot about modal logic, which strikes me as a good balance between using a powerful type of logic, as well as being pretty realistic to implement with reasonable performance.
Although I feel like I have only dipped my toes into the subject, and there is a lot more depths that I have yet to explore.

I also learned a lot about the programming language Zig, which has been a hobby-language of mine for quite a while, but I have not been able to test it on a project of this scale.
Generally I would say that if you don't mind having to take responsibility for the memory, it is a language which gives a lot of power to a programmer who wants to experiment with data-representation and data structures in general.
Some aspects that would have been great to look into if I had more time would be multi-core programming using Zig, in order to optimize queries and even generation of the possible worlds.

\todo[inline]{Maybe this should go in further work rather than here}
Of course this method focuses on what can certainly be known, so if index 2 on the hand cannot be "playable" in all possible worlds, then it will not be played, but it is not unthinkable to use the possible worlds as a sort of probability, and if most of these worlds, we find that we can play index 2, then it is probably successful idea.
But I will focus on what an agent can know with certitude and therefore I define a card to be "playable" if it is playable in all possible worlds.


